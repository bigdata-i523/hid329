\documentclass[sigconf]{acmart}

\usepackage{hyperref}

\usepackage{endfloat}
\renewcommand{\efloatseparator}{\mbox{}} % no new page between figures

\usepackage{booktabs} % For formal tables

\settopmatter{printacmref=false} % Removes citation information below abstract
\renewcommand\footnotetextcopyrightpermission[1]{} % removes footnote with conference information in first column
\pagestyle{plain} % removes running headers

\begin{document}
\title{Big Data in Higher Education Marketing}


\author{Ashley Miller}
\orcid{HID329}
\affiliation{%
  \institution{Indiana University}
}
\email{admille@iu.edu}



% The default list of authors is too long for headers}
\renewcommand{\shortauthors}{B. Trovato et al.}


\begin{abstract}
While the collection of vast amounts of data in the world higher education has happened for decades, the use of big data applications and analytics is fairly new to this environment. There is a need to understand how the use of big data can help higher education further understand student needs as well as stay relevant in a digital and evolving age of technological advances, tools, and skills. Further, the population of students going to college is on the decline which increases competition and the need for institutions to be more strategic in their efforts for attracting students to their institutions. This purpose of this paper is to explore at a very high level how higher education could utilize big data to inform marketing initiatives in recruiting and enrolling students. 
\end{abstract}

\keywords{Big Data, Higher Education, Marketing, Analytics \LaTeX}


\maketitle

\section{Introduction}

Today’s colleges and universities are drowning in data. With the emergence of big data, institutions are now faced with providing useful analysis and reports to a variety of stakeholders including administrators, professors, as well as to the students themselves. A variety of challenges lie in the path of institutions using big data effectively such as finding the necessary skill set for staff, technology tools and resources, as well as understanding then what to do with the data collected to better inform decision-making. 

While there is literature that addresses utilizing big data for learning analytics and even course enrollment and development, as Daniel states, there is still “limited research into big data in higher education” (2015).  This paper seeks to explore ways in which higher education could utilize big data in their marketing efforts for recruiting and enrolling students as well as what gaps may still exist in the quest to understand today’s college search as they make their choice on which university to attend. 


\section{Current Environment}

While high school graduation rates have increased over time, the number of those who go on to pursue higher education has been on the decline for the past four years (The Atlantic, 2016). Meanwhile, the number of four-year institutions in the United States has increased with there now being more than 3,000 college options (NCES, 2014). Increased competition and fewer students have made the higher education marketplace crowded and convoluted. There are a variety of factors that go into a student’s decision on where to attend and ultimately what area to study. In their 2013 trends report, the Lawlor group identified a number of aspects that will impact the higher education landscape, among those included are:

\begin{itemize}
\item Demographics of today’s college student is changing with more women attending college than men in addition to an increase in ethnic and socio-economic diversity as well as first-generation students. 

\item The college search process today happens primarily in the digital space which include third-party websites, email, social media, and digital advertising. This ‘Generation Z’ of student grew up in a technology rich and connected environment which means that colleges have to also be ‘constantly on’ in their effort to recruit and enroll students.

\item The need to showcase the ‘value’ of going to college, not only through the quality of education received relative to the price paid but also through outcomes-level data including placement rates and starting salaries of recent graduates. 
\end{itemize}

\section{Big Data to Segment Students by Demographics}
With these trends in mind, there is a need for institutions to more effectively target and recruit students. Big data can be one way to better inform these efforts and also help with the return-on-investment (ROI) for advertising and marketing related efforts. Other universities have capitalized on utilizing big data in attracting students. For instance, St. Louis University described a process of retroactively looking at demographics of students who succeeded at the university and had high satisfaction scores (The Atlantic 2017). This information coupled with nearly “120 data points” gave insight to the admissions team when exploring new markets as well as identified clusters of students that may be interested in attending St. Louis University. The university was then able to develop a targeted digital campaign in these areas that they believed include students who would be a ‘good fit’. With the reliance on big data, the university was able to reduce costs as the need to mass market went away and ultimately increased enrollment as a result.  

\section{Big Data to Understand Student Behavior Online}
The web environment is common tool in college exploration as a report by Ruffalo Noel Levitz shows that three out of four high school students state that the institution’s website is their most used resource when exploring colleges (2016). Web analytics provides a wealth of information on users such as how much time is spent on certain pages, bounce rate, paths in website exploration and ultimately conversion rates when various goals are completed such as scheduling a tour or filling out an application for college. Google Analytics is one tool used to track and evaluate efforts on websites. Higher education institutions could take advantage of this tool by tracking top pages viewed, geography and age of visitors, as well as areas where they may be ‘losing’ students in the information search process. With this data, institutions can identify opportunities for improvement in ensuring students are finding the information they need in a timely and efficient way as well as develop customized marketing efforts to invite students back into the experience to complete various calls-to-action. 

\section{Big Data to Convey Value}
Utilizing big data to understand outcomes of students can help tell the value story to prospective students. By tracking the experiences among currently students during their four (or more) year college career, predictive analytics could be implemented to determine which combination set of experiences best contribute to the ‘success’ of  a student. One university to showcase the impact of this data on outcomes is American University with their ‘We Know Success’ tool (CITE SOURCE). By collecting data from graduates over time, the university can further showcase to others 








\section{Conclusions}

Competition for today’s student will only increase with changing educational needs and offerings, including development of emerging degree programs as well as delivery, including online classes. In order for the use of big data in higher education marketing to be successful, there are basic measures that have to be met. RAND outlines some key considerations when using big data for effective decision-making which include: accessibility, quality, timeliness, and motivation to use. 

For marketers in higher education, they need to have access to necessary data about current as well as prospective students to better tailor messaging and marketing efforts appropriately. With this, the validity of available data is key as making decisions based on ‘bad’ or incomplete data can be problematic and costly for an institution. Given the nature of the web environment that is constantly changing, obtaining data in a timely manner is crucial so action can be taken at the right time. Further, there has to be a culture within an institution that motivates others to make data-driven decisions.   




\bibliographystyle{ACM-Reference-Format}
\bibliography{report} 

\end{document}
